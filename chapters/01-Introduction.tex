%!TEX root = ../main.tex

\chapter{Introduction}\label{cha:introduction}

Deep Learning is a subclass of Machine Learning that has achieved state of the art results in a wide variety of domains.
They were originally proposed by McCulloch and Pitts in 1943 as a model for how the brain processes information \cite{McCulloch_Pitts_1943}.
In 1957, Rosenblatt created the Perceptron \cite{Rosenblatt_1957, Rosenblatt_1958}, however it was shown that it is impossible for these to learn the simple XOR function \cite{Minsky_Papert_1988}.
In the 1980's neural networks became incredibly popular again.
Rumelhart, Hinton and Williams wrote a particularly famous paper outlining the backpropagation algorithm for training networks of neuron-like units \cite{Rumelhart_Hinton_Williams_1986}.
This algorithm remains at the core of all deep learning today.


\section{Neural Networks}\label{sec:neural_network_int}
A neural network is a collection of connected nodes (simplified versions of  biological neurons).
A connection between two nodes (equivalent of a biological synapse) can transfer data from one node to the next.
The signal processed between two nodes is a real valued number, and the output is a non-linear function of the sum of its inputs.

\section{Problem Description}\label{sec:problem_description_int}
%TODO introduction to deep learning in healthcare
Detecting cancers in medical images is an important task in modern health care.

\subsection{PET Scans}\label{subsec:pet_scan_int}
Positron Emission Tomography (PET) scans are used to produce 3 Dimensional images of the human body and are often combined with CT scans (called PET-CT) or MRI scans (PET-MRI) in order to increase the level of detail.
PET scans are particularly useful because they describe how certain parts of the body are functioning in addition to their physical appearance.
PET scans are most commonly used for investigating confirmed cases of cancer to determine how far the cancer has spread and how well it's responding to treatment \cite{Radiology_ACR, PET_scan}.
The procedure is as follows:

\begin{itemize}
    \item A radioactive tracer (often glucose based) is introduced to the patient either by injection into the bloodstream, swallowing or inhalation as a gas.
    \item For a glucose based tracer, cells in the body with a high metabolic rate will accumulate the tracer.
    \item The tracer gives off a small amount of energy in the form of gamma rays which is detected by the scanner.
    \item A computer then produces a 3D image in which hotspots indicate a large amount of tracer accumulation and therefore a high level of metabolic activity.
\end{itemize}

% TODO include some example pet scan images

\subsection{Current Data}\label{subsec:current_data_intro}

% %TODO explain The Data
% %TODO explain Ground Truth and what we hope to achieve


\section{Structure}\label{sec:structure}

\begin{itemize}
    \item In Chapter \ref{cha:literature} an overview of previous literature regarding Deep Learning is given.
    \item In Chapter \ref{cha:initial_results} an outline of current work and potential future developments is outlined.
\end{itemize}