%!TEX root = ../main.tex

\chapter{Introduction}\label{cha:introduction}

Deep Learning is a subclass of Machine Learning that has achieved state of the art results in a wide variety of domains.

\section{Deep Learning}\label{sec:deep_learning_int}
% overview of what deep learning actually is

\section{Problem Description}\label{sec:problem_description_int}
% The Data
% PET scan explanation
% Ground Truth and what we hope to achieve
Detecting cancers in medical images is an important task in modern health care.

Positron Emission Tomography (PET) scans are used to produce 3 Dimensional images of the human body and are often combined with CT scans (called PET-CT) or MRI scans (PET-MRI) in order to increase the level of detail.
PET scans are particularly useful because they describe how certain parts of the body are functioning in addition to their physical appearance.
PET scans are most commonly used for investigating confirmed cases of cancer to determine how far the cancer has spread and how well it's responding to treatment \cite{Radiology_ACR, PET_scan}.
The procedure is as follows:

\begin{itemize}
    \item A radioactive tracer (often glucose based) is introduced to the patient either by injection into the bloodstream, swallowing or inhalation as a gas.
    \item For a glucose based tracer, cells in the body with a high metabolic rate will accumulate the tracer.
    \item The tracer gives off a small amount of energy in the form of gamma rays which is detected by the scanner.
    \item A computer then produces a 3D image in which hotspots indicate a large amount of tracer accumulation and therefore a high level of metabolic activity.
\end{itemize}



\section{Structure}\label{sec:structure}

\begin{itemize}
    \item In Chapter \ref{cha:literature} an overview of previous literature regarding Deep Learning is given.
    \item In Chapter \ref{cha:initial_results} an outline of current work and potential future developments is outlined.
\end{itemize}